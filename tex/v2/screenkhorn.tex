\documentclass{article}

% if you need to pass options to natbib, use, e.g.:
% \PassOptionsToPackage{numbers, compress}{natbib}
% before loading neurips_2019

% ready for submission
% \usepackage{neurips_2019}

% to compile a preprint version, e.g., for submission to arXiv, add add the
% [preprint] option:
%\usepackage[preprint]{neurips_2019}

% to compile a camera-ready version, add the [final] option, e.g.:
% \usepackage[preprint]{neurips_2019}
\usepackage[]{neurips_2019}
% to avoid loading the natbib package, add option nonatbib:
% \usepackage[nonatbib]{neurips_2019}

\usepackage[utf8]{inputenc} % allow utf-8 input
\usepackage[T1]{fontenc}    % use 8-bit T1 fonts
\usepackage{hyperref}       % hyperlinks
\usepackage{url}            % simple URL typesetting
\usepackage{booktabs}       % professional-quality tables
\usepackage{amsfonts}       % blackboard math symbols
\usepackage{nicefrac}       % compact symbols for 1/2, etc.
\usepackage{microtype}      % microtypography

%--------------------------------------------------------------------------------------------------------------------------------% 

\RequirePackage{mathrsfs, amsthm, amsmath, amsfonts, amssymb, mathtools}%
\usepackage[titletoc,title]{appendix}%
\usepackage{dsfont}%
\usepackage{caption} % 
\usepackage{subcaption}
\usepackage{float}%
\usepackage{graphicx, color}% 
\usepackage{bm}%bold equations

%% formatting appendices
\usepackage{etoolbox}
\patchcmd{\appendices}{\quad}{. }{}{}

%% tables
\usepackage{pgfplots}%
\usepackage{booktabs,multirow,array,multicol}
\newcommand{\otoprule}{\midrule[\heavyrulewidth]}

%% algorithms
\usepackage[linesnumbered,ruled,vlined]{algorithm2e}

%% sections
\newtheorem{theorem}{Theorem}
\newtheorem{corollary}{Corollary}
\newtheorem{lemma}{Lemma}
\newtheorem{fact}{Fact}
\newtheorem{proposition}{Proposition}
\newtheorem{definition}{Definition}%[section]
\newtheorem*{notation}{Notation}%[section]
\newtheorem{discus}{Discussion}%[section]
\newtheorem{remark}{Remark}%[section]
\newtheorem{example}{Example}%[section]
\newtheorem{exs}{Examples}%[section]
\newtheorem{ca}{Cas}
\newtheorem{remarks}{Remarks}%[section]
\newtheorem{assumption}{Assumption}{\bf}{\rm}%

% math macros

\usepackage{relsize}
\newcommand*{\defeq}{\stackrel{\mathsmaller{\mathsf{def.}}}{=}}

\newcommand{\ba}{\mathbf{a}}%
\newcommand{\bb}{\mathbf{b}}%

\newcommand{\bC}{{\mathbf C}}%
\newcommand{\bP}{{\mathbf P}}%
\newcommand{\bU}{{\mathbf U}}%

\newcommand{\cX}{\mathcal X}%
\newcommand{\cY}{\mathcal Y}%
\newcommand{\cM}{\mathcal M}%
\newcommand{\cC}{\mathcal C}%

\newcommand{\td}{\text d}%
\newcommand*{\argminl}{\argmin\limits}%
\newcommand{\bigO}{\mathcal{O}}
\newcommand{\bs}{\boldsymbol}
\newcommand{\sumin}{\sum_{i=1}^n}
\newcommand{\sumip}{\sum_{j=1}^m}
\newcommand{\intt}{\int_{0}^{t}}
\newcommand{\inttau}{\int_{0}^{\gamma}}
\newcommand{\I}{1\!\mbox{I}}
\newcommand{\e}{{\mathrm e}}
\newcommand{\diff}{\mathrm d}
\newcommand{\noi}{\noindent}
\newcommand{\eps}{\varepsilon}
\usepackage{datenumber}
\newcommand{\sptr}[2]{\langle #1 | #2 \rangle}
\newcommand{\inr}[1]{\langle #1 \rangle}
\newcommand{\inrbig}[1]{\bigg\langle #1 \bigg\rangle}
\newcommand{\ind}[1]{{\mathds{1}}_{{#1}}}%
\newcommand{\norm}[1]{\|#1\|}
\newcommand{\R}{{\mathbb{R}}}
\newcommand{\E}{\mathds{E}}
\newcommand{\V}{\mathds{V}}
\newcommand{\Lim}{\displaystyle \lim\limits}
\newcommand{\overbar}[1]{\mkern 1.5mu\overline{\mkern-1.5mu#1\mkern-1.5mu}\mkern 1.5mu}
\newcommand{\cD}{\mathcal D}
\newcommand{\cE}{\mathcal E}
\newcommand{\cT}{\mathcal T}
\newcommand{\bA}{{\boldsymbol A}}%
\newcommand{\bcA}{\boldsymbol{\mathcal{C}}}%
\newcommand{\bB}{{\boldsymbol B}}%
\newcommand{\bcB}{\boldsymbol{\mathcal{B}}}%
\newcommand{\bD}{{\boldsymbol D}}%
\newcommand{\bE}{{\boldsymbol E}}%
\newcommand{\bF}{{\boldsymbol F}}%
\newcommand{\bG}{{\boldsymbol G}}%
\newcommand{\bH}{{\boldsymbol H}}%
\newcommand{\bI}{{\boldsymbol I}}%
\newcommand{\bJ}{{\boldsymbol J}}%
\newcommand{\bK}{{\boldsymbol K}}%
\newcommand{\bL}{{\boldsymbol L}}%
\newcommand{\bM}{{\boldsymbol M}}%
\newcommand{\bN}{{\boldsymbol N}}%
\newcommand{\bO}{\boldsymbol 0}%
\newcommand{\bQ}{\boldsymbol Q}%
\newcommand{\bR}{\boldsymbol R}%
\newcommand{\cS}{{\mathcal{S}}}%
\newcommand{\bcQ}{\boldsymbol{\mathcal{Q}}}%
\newcommand{\bV}{\boldsymbol V}%
\newcommand{\bW}{\boldsymbol W}%
\newcommand{\bX}{\boldsymbol  X}%
\newcommand{\bT}{\boldsymbol  T}%
\newcommand{\bcE}{\boldsymbol{\mathcal{E}}}%
\newcommand{\bcH}{\boldsymbol{\mathcal{H}}}%
\newcommand{\bcX}{\boldsymbol{\mathcal{X}}}%
\newcommand{\bcJ}{\boldsymbol{\mathcal{J}}}%
\newcommand{\bcY}{\boldsymbol{\mathcal{Y}}}%
\newcommand{\bcW}{\boldsymbol{\mathcal{W}}}%
\newcommand{\bcT}{\boldsymbol{\mathcal{T}}}%
\newcommand{\bcM}{\boldsymbol{\mathcal{M}}}%
\newcommand{\bcZ}{\boldsymbol{\mathcal{Z}}}%
\newcommand{\bcO}{\boldsymbol{\mathcal{O}}}%
\newcommand{\bcG}{\boldsymbol{\mathcal{G}}}%
\newcommand{\bcU}{\boldsymbol{\mathcal{U}}}%
\newcommand{\bcV}{\boldsymbol{\mathcal{V}}}%
\newcommand{\bcR}{\boldsymbol{\mathcal{R}}}%
\newcommand{\bcI}{\boldsymbol{\mathcal{I}}}%
\newcommand{\by}{\boldsymbol y}%
\newcommand{\bu}{\boldsymbol u}%
\newcommand{\bv}{\boldsymbol  v}%
\newcommand{\bz}{\boldsymbol z}%
\newcommand{\bY}{\boldsymbol{Y}}%
\newcommand{\bZ}{\boldsymbol Z}
\newcommand{\bg}{\boldsymbol g}%
\newcommand{\beps}{\boldsymbol \varepsilon}%
\newcommand{\bSigma}{\boldsymbol \Sigma}%
\newcommand{\blambda}{\boldsymbol  \lambda}
\newcommand{\bbeta}{\boldsymbol  \beta}%
\newcommand{\bLambda}{\boldsymbol  \Lambda}%
\newcommand{\bTheta}{\boldsymbol  \Theta}
\newcommand{\bDelta}{\boldsymbol{\Delta}}%
\newcommand{\bXi}{\boldsymbol{\Xi}}%
\newcommand{\ov}{\overrightarrow}%
\newcommand{\cP}{\mathcal P}%
\newcommand{\cQ}{\mathcal Q}%
\renewcommand{\P}{\mathds{P}}
\newcommand{\bctheta}{\boldsymbol{\mathcal{\Theta}}}%
\newcommand{\btheta}{\boldsymbol  \Theta}%
\newcommand{\bcdot}{\raisebox{-0.80ex}{\scalebox{1.8}{$\cdot$}}}
\newcommand{\cst}{\raisebox{-0.15ex}{\scalebox{1.30}{$c$}}}
\newcommand{\Cst}{\raisebox{-0.15ex}{\scalebox{1.10}{$C$}}}
\newcommand{\varsig}{\raisebox{-0.15ex}{\scalebox{1.30}{$\varsigma$}}}
\usepackage{accents}
\newcommand*{\dt}[1]{\accentset{\mbox{\large\bfseries .}}{#1}}
%{\accentset{\raisebox{0ex}{$.$}}{#1}}
\newcommand*{\ddt}[1]{\accentset{\raisebox{0ex}{$\,\,\star$}}{#1}}

\newcommand*{\fstderiv}[1]{\accentset{\mbox{\Large\bfseries .}}{#1}}
\newcommand*{\scdderiv}[1]{\accentset{\mbox{\Large\bfseries .\hspace{-0.25ex}.}}{#1}}

%% math operators
\DeclareMathOperator*{\argmin}{\arg\!\min}
\DeclareMathOperator*{\argmax}{\arg\!\max}

\DeclarePairedDelimiter\ceil{\lceil}{\rceil}
\DeclarePairedDelimiter\floor{\lfloor}{\rfloor}

% todo
\usepackage[colorinlistoftodos,prependcaption,textsize=tiny]{todonotes}
%--------------------------------------------------------------------------------------------------------------------------------%

\title{Screening Sinkhorn Algorithm via Dual Projections}

\begin{document}

\maketitle

\begin{abstract}

This paper deals with the problem of approximating optimal transport (OT) distance between two discrete measures.
Our proposed approach involves a convex projection of the \emph{dual of Sinkhorn divergence}, allowing to formulate two appropriate active indices sets for the potential variables.
These indices sets depend on two parameters acting like a threshold and a scaling factor and they are both directly linked to a priori fixed number budget of points from the supports of the given discrete measures.
This new analysis induces a screened version of the dual of Sinkhorn divergence and suggests the \emph{Screenkhorn} algorithm.
We illustrate the favorable performance of Screenkhorn in practice with numerical experiments on synthetic and real datasets.
\end{abstract}

\section{Introduction}

Computing OT distances between pairs of probability measures or histograms, such as the earth mover's distance~\cite{werman1985,Rubner2000} and Monge-Kantorovich or Wasserstein distance~\cite{villani09optimal}, are currently generating an increasing attraction in different machine learning tasks~\cite{pmlr-v32-solomon14,kusnerb2015,pmlr-v70-arjovsky17a,ho2017}, statistics~\cite{frogner2015nips,panaretos2016,ebert2017ConstructionON,bigot2017,flamary2018WDA}, and computer vision~\cite{bonnel2011,Rubner2000,solomon2015}, among other applications~\cite{klouri17,peyre2019COTnowpublisher}.
In many of these problems, OT exploits the geometric features of the objects at hand in the underlying spaces to be leveraged in comparing probability measures.
This effectively leads to improve performance of methods that are oblivious to the geometry, for example the chi-squared distances or the Kullback-Leibler divergence.
Unfortunately, this advantage comes at the price of an enormous computational cost of solving the OT problem, that can be prohibitive in large scale applications.
For instance, the OT between two histograms with supports of equal size $n$ can be formulated as a linear programming problem that requires generally super $\bigO(n^3)$~\cite{pele2009} arithmetic operations, which is problematic when $n$ is larger than $10^3.$

A remedy to the heavy computation burden of OT lies in a prevalent approach referred to as regularized OT~\cite{cuturinips13} and operates by adding an entropic regularization penalty to the original problem.  
Such a regularization guarantees a unique solution, since the objective function is strongly convex, and a greater computational stability.
Furthermore,~\cite{cuturinips13} proposed the so-called {dual of Sinkhorn divergence} as the dual of the entropic problem and noticed that finding the dual solution was equivalent to finding two diagonal matrices that made a full matrix bistochastic.
Therefore, the OT can be solved efficiently with celebrated matrix scaling algorithms, such as Sinkhorn's fixed point iteration method~\cite{sinkhorn1967,knight2008,kalantari2008}. 

% \paragraph{Related work.} 

Sinkhorn scaling for computing OT distances is a well studied problem in many recent works. 
The main idea is to improve the matrix-vector operations that are the true computational bottleneck of Sinkhorn’s algorithm. 
\cite{altschulernips17} proposed the Greenkhorn algorithm, a greedy version of Sinkhorn algorithm that selects columns and rows to be updated that most violate the constraints.
\cite{altschuler2018Nystrom} provided the Nys-Sink algorithm which is based on low-rank approximation of the cost matrix using Nystrom method.
Other classical optimization algorithms have been considered to approximate the OT, for instance accelerated gradient descent~\cite{dvurechensky18aICML,lin2019}, quasi Newton methods~\cite{blondel2018ICML,cuturi2016SIAM} and stochastic gradient descent~\cite{genevay2016stochOT,khalilabid2018}. 

% \paragraph{Our contribution.}

We give a new algorithm to approximate the regularized OT distance between discrete measures. 
Our algorithmic analysis is based on an approximate of the dual of Sinkhorn divergence by adding new constraints feasibility. 
These constraints are defined through a convex set which depends on two parameters, acting like threshold and scaling factor.
We prove that dual solution of this approximation guarantees the existence of two active indices sets for the potential variables.
These active sets are both directly linked to a priori fixed number budget of points from the supports of the given discrete measures.
We then restrict the constraints feasibility with respect to the active sets to get a ``screened'' version of the dual of Sinkhorn divergence. 
The Screenkhorn algorithm developed in this paper relies on two steps; the first one consists of an initialization step devoted to determine the active sets while the second is a constrained L-BFGS solver~\cite{zhu1997-LBFGS,cuturi2016SIAM,blondel2018ICML}. 


% \paragraph{Outline of paper.} 

The remainder of the paper is organized as follow. In Section~\ref{sec:optimal_transport} we briefly review the basic setup of regularized discrete OT. 
Section~\ref{sec:screenkhorn} contains our main contribution, that is, the Screenkhorn algorithm. 
Section~\ref{sec:error_analysis} devotes to theoretical guarantees for the marginal violations of Screenkhorn. 
In Section~\ref{sec:numerical_experiments} we present numerical results for the proposed algorithm, compared with the state-of-art Sinkhorn algorithm as implemented in~\cite{flamary2017pot}. 
The proofs of theoretical results are postponed to the supplementary material.


\paragraph{Notation.}

For any positive matrix $T \in \R^{n\times m}$, we define its negative entropy as $H(T) = -\sum_{i,j} T_{ij} \log(T_{ij}).$
Let $r(T) = T\mathbf 1_m \in \R^n$ and $c(T) = T^\top\mathbf 1_n \in \R^m$ denote the rows and columns sums of $T$ respectively. The coordinates $r_i(T)$ and $c_j(T)$ denote the $i$-th row sum and the $j$-th column sum of $T$, respectively.
The scalar product between two matrices denotes the usual inner product, that is $\inr{T, W} = \text{tr}(T^\top W) = \sum_{i,j}T_{ij}W_{ij},$ where $T^\top$ is the transpose of $T$. 
We write $\mathbf{1}$ (resp. $\mathbf{0}$) the vector having all coordinates equal to one (resp. zero).
$\Delta(w)$ denotes the diag operator, such that if $w \in \R^n$, then $\Delta(w) = \text{diag}(w_1, \ldots, w_n)\in \R^{n\times n}$.
For a set of indices $L=\{i_1, \ldots, i_k\} \subseteq \{1, \ldots, n\}$ satisfying $i_1 < \cdots <i_k,$ we denote the complementary set of $L$ by $L^\complement = \{1, \ldots, n\} \backslash L$. We also denote $|L|$ the cardinality of $L$.
Given a vector $w \in \R^n$, we denote $w_L= (w_{i_1}, \ldots, w_{i_k})^\top \in \R^k$ and its complementary $w_{L^\complement} \in \R^{n- k}$.  The notation is similar for matrices; given another subset of indices $S = \{j_1, \ldots, j_l\} \subseteq \{1, \ldots, m\}$ with $j_1 < \cdots <j_l,$ and a matrix $T\in \R^{n\times m}$, we use $T_{(L,S)}$, to denote the submatrix of $T$, namely the rows and columns of $T_{(L,S)}$ are indexed by $L$ and $S$ respectively.
When applied to matrices and vectors,  $\odot$ and $\oslash$ (Hadamard product and division) and exponential notations refer to elementwise operators.
Given two real numbers $a$ and $b$, we write $a\vee b = \max(a,b)$ and $a\wedge b = \min(a,b).$

\section{Regularized discrete OT} 
\label{sec:optimal_transport}

 We briefly present in this section the setup of OT between two discrete measures. We then consider the case when those distributions are only available through a finite number of samples, that is $\mu = \sum_{i=1}^n \mu_i \delta_{x_i} \in \Sigma_n$ and $\nu = \sum_{j=1}^m \nu_i \delta_{y_j} \in \Sigma_m$, where $\Sigma_n$ is the probability simplex with $n$ bins, namely the set of probability vectors in $\R_+^n$, i.e., $\Sigma_n = \{w \in \R_+^n: \sum_{i=1}^n w_i = 1\}.$
We denote their probabilistic couplings set as $\Pi(\mu, \nu) = \{P \in \R_+^{n\times m}, P\mathbf{1}_m = \mu, P^\top \mathbf{1}_n = \nu\}.$ 
\paragraph{Sinkhorn divergence.}

Approximating OT distance between the two discrete measures $\mu$ and $\nu$  amounts to solving a linear problem~\cite{kantorovich1942} given by
\begin{equation}
  \label{monge-kantorovich}
  \mathcal{S}(\mu, \nu) =  \min_{P\in \Pi(\mu, \nu)} \inr{C, P},
\end{equation}
where $P= (P_{ij}) \in \R^{n\times m}$ is called the transportation plan, namely each entry $P_{ij}$ represents the fraction of mass moving from $x_i$ to $y_j$, and $C= (C_{ij}) \in \R^{n\times m}$ is a cost matrix comprised of nonnegative elements and related to the energy needed to move a probability mass from $x_i$ to $y_j$. 
The entropic regularization of OT distances~\citep{cuturinips13} relies on the addition of a penalty term as follows:
\begin{equation}
\label{sinkhorn-primal}
  \mathcal{S}_\eta(\mu, \nu) =  \min_{P\in \Pi(\mu, \nu)} \{\inr{C, P} - \eta H(P)\},
\end{equation}
where $\eta > 0$ is a regularization parameter. We refer to $\mathcal{S}_\eta(\mu, \nu) $ as the \emph{Sinkhorn divergence}~\citep{cuturinips13}.

\paragraph{Dual of Sinkhorn divergence.}

Below we provide the derivation of the dual problem for the regularized OT problem~\eqref{sinkhorn-primal}. Towards this end, we begin with writing its Lagrangian dual function :
\begin{equation*}
  \mathscr{L}(P,y, z) = \inr{C,P} + \eta \inr{\log P, P} + \inr{y, P\mathbf{1}_m - \mu} + \inr{z,P^\top \mathbf{1}_n - \nu}.
\end{equation*}
The dual of Sinkhorn divergence can be derived by solving $\min_{P \in \R_+^{n\times m}}\mathscr{L}(P,y, z)$. It is easy to check that objective function $P\mapsto \mathscr{L}(P,y, z)$ is strongly convex and differentiable. Hence, one can solve the latter minimum by setting $\nabla_P \mathscr{L}(P,y, z)$ to $\mathbf{0}_{n\times m}$. Therefore, we get 
\begin{equation}
  P^\star_{ij} = \exp\Big(- \frac{1}{\eta} (y_i + z_j + C_{ij}) - 1\Big), 
\end{equation}
for all $i=1, \ldots, n$ and $j=1, \ldots, m$. Plugging this solution,  and setting the change of variables $u = -y/\eta - 1/2$ and $v = - z/\eta - 1/2$, the dual problem is given by
\begin{equation}
\label{sinkhorn-dual}
\min_{u \in \R^n, v\in\R^m}\big\{\Psi(u,v):= \mathbf{1}_n^\top B(u,v)\mathbf{1}_m - \inr{u, \mu} - \inr{v, \nu} \big\},
\end{equation}
where $B(u,v) = \Delta(e^{u}) K \Delta(e^{v})$ and $K = e^{-C/\eta}$ stands for the Gibbs kernel associated to the cost matrix $C$. 
We refer to problem~\eqref{sinkhorn-dual} to the \emph{dual of Sinkhorn divergence}.
Therefore, the optimal solution $P^\star$ of Sinkhorn divergence takes the form $P^\star = \Delta(e^{u^\star}) K \Delta(e^{v^\star})$
where the couple $(u^\star, v^\star)$ satisfies:
\begin{align*}
\label{sinkhorn-dual}
  (u^\star, v^\star) &= \argmin_{u \in \R^{n}, v\in \R^m} \{\Psi(u,v)\}.
\end{align*}
Note that the matrices $\Delta(e^{u^\star})$ and $\Delta(e^{v^\star})$ are unique up to a constant factor~\citep{sinkhorn1967}. Furthermore, $P^\star$ can be solved efficiently by iterative Bregman projections~\cite{benamou2015IterativeBP} referred as Sinkhorn iterations, and the method is referred as Sinkhorn algorithm which, recently, is proven to achieve a near-$\bigO(n^2)$ complexity.

\section{Screened dual of Sinkhorn divergence}
\label{sec:screenkhorn}

% In this section we describe the main algorithm studied in this paper. 
For a fixed $\varepsilon > 0$ and $\kappa > 0$ we define an \emph{approximate dual of Sinkhorn divergence} as follows:
\begin{equation} 
\label{screen-sinkhorn}
\min_{u \in \mathcal{C}^n_{\frac \varepsilon \kappa}, v\in \mathcal{C}^m_{\varepsilon\kappa}} \big\{\Psi_{\kappa}(u,v):= \mathbf{1}_n^\top B(u,v)\mathbf{1}_m - \inr{\kappa u, \mu} - \inr{\frac v\kappa, \nu} \big\},
\end{equation}
where $\mathcal{C}^r_{\alpha} \subseteq \R^r$, for $r\in \mathbb N$ and $\alpha >0$, is a convex set  given by $\mathcal{C}^r_{\alpha} = \{w \in \R^{r}: \min_{1\leq i\leq r} e^{w_i} \geq \alpha\}$.

The objective function $\Psi_{\kappa}$ is convex with respect to $(u,v)$, then the set of optima of problem~\eqref{screen-sinkhorn} is non empty. 
The $\kappa$-parameter plays a role of scaling factor, namely it allows to get a closed order of the potential variables $e^u$ and $e^v$, while the $\varepsilon$-parameter acts like a threshold for $e^u$ and $e^v$.
Note that the approximate dual of Sinkhorn divergence coincides with the dual of Sinkhorn divergence~\eqref{sinkhorn-dual} in the setting of $\varepsilon=0$ and $\kappa=1$.
The screening procedure presented in this work is based on constructing two {active sets} $I_{\varepsilon, \kappa}$ and $J_{\varepsilon, \kappa}$ throughout the dual problem of~\eqref{screen-sinkhorn} in the following way:

\begin{lemma}
\label{lemma_actives_sets}
Let $(u^{*}, v^{*})$ be an optimal solution of the problem~\eqref{screen-sinkhorn}. 
Define
\begin{equation}
\label{I_epsilon_kappa_J_epsilon_kappa}
I_{\varepsilon,\kappa} = \big\{i=1, \ldots, n: \mu_i \geq \frac {\varepsilon^2} \kappa^{} r_i(K)\big\}, J_{\varepsilon,\kappa} = \big\{j=1, \ldots, m: \nu_j \geq \kappa{\varepsilon^2}{} c_j(K)\big\}
\end{equation}
Then one has $e^{u^{*}_i} = \varepsilon\kappa^{-1}$ and $e^{v^{*}_j} = \varepsilon\kappa$ for all $i \in I^\complement_{\varepsilon,\kappa} $ and $j\in J^\complement_{\varepsilon,\kappa} .$
\end{lemma}

First order conditions applied to $(u^{*}, v^{*})$ ensure that if $e^{u^{*}_i} > \varepsilon\kappa^{-1}$ then $e^{u^{*}_i} (Ke^{v^{*}})_i =  \kappa\mu_i$ and if $e^{v^{*}_j} > \varepsilon\kappa$ then $e^{v^{*}_j} (K^\top e^{u^{*}})_j =  \kappa^{-1}\nu_j$ 
which correspond to the Sinkhorn marginal conditions up to the scaling factor $\kappa$. 

\paragraph{Screening with a fixed number budget of points.}

Recall that the approximate dual of Sinkhorn divergence is defined with respect to $\varepsilon$ and $\kappa$. 
The explicit determination of its values depends on a priori \emph{fixed number budget of points} from the supports of $\mu$ and $\nu$. % in problem~\eqref{screen-sinkhorn}.  
In the sequel of the paper, we denote by $n_b \in\{1, \ldots, n\}$ and the $m_b\in\{1, \ldots, m\}$ the number budget of points to be given for resolving problem~\eqref{screen-sinkhorn}. 

Let us define $\xi \in \R^n$ and $\zeta \in \R^m$ to be the ordered decreasing vectors of $\mu \oslash r(K)$ and $\nu \oslash c(K)$ respectively, that is $\xi_1 \geq \xi_2 \geq \cdots \geq \xi_n$ and $\zeta_1 \geq \zeta_2 \geq \cdots \geq \zeta_m$.
To keep only $n_b$-budget and $m_b$-budget of points, the parameters $\kappa$ and $\varepsilon$ satisfy ${\varepsilon^2}\kappa^{-1} = \xi_{n_b}$ and $\varepsilon^2\kappa = \zeta_{m_b}$. Hence 
\begin{equation}
\label{epsilon_kappa}
 \varepsilon = (\xi_{n_b}\zeta_{m_b})^{1/4} \text{ and } \kappa = \sqrt{\frac{\zeta_{m_b}}{\xi_{n_b}}}.
\end{equation}
Note that $|I_{\varepsilon, \kappa}| = n_b$ and $|J_{\varepsilon, \kappa}| = m_b$. 
Using the previous analysis, any solution $(u^*, v^*)$ of problem~\eqref{screen-sinkhorn} satisfy $e^{u^*_i} \geq \varepsilon\kappa^{-1}$ and $e^{v^*_j} \geq \varepsilon\kappa$ for all $(i,j) \in (I_{\varepsilon,\kappa}\times J_{\varepsilon,\kappa}),$ and $e^{u^*_i} = \varepsilon\kappa^{-1}$ and $e^{v^*_j} = \varepsilon\kappa$ for all $(i,j) \in (I^\complement_{\varepsilon,\kappa}\times J^\complement_{\varepsilon,\kappa})$.

Basing on that facts we restrict the constraints feasibility $\mathcal{C}^n_{\frac \varepsilon \kappa} \cap \mathcal{C}^m_{\varepsilon\kappa}$ in problem~\eqref{screen-sinkhorn} to the screened domain $\mathcal{U}_{\text{sc}} \cap \mathcal{V}_{\text{sc}}$ where 
\begin{equation*}
\mathcal{U}_{\text{sc}} = \{u \in \R^n: e^{u_{I_{\varepsilon,\kappa}}} \succeq \frac \varepsilon\kappa\mathbf 1_{n_b}, \text{ and } e^{u_{I^\complement_{\varepsilon,\kappa}}} = \frac\varepsilon\kappa\mathbf 1_{n - n_b}\},
\end{equation*}
and 
\begin{equation*}
	\mathcal{V}_{\text{sc}} =\{v\in\R^m: e^{v_{J_{\varepsilon,\kappa}}} \succeq \varepsilon\kappa \mathbf{1}_{m_b}, \text{ and } e^{v_{J^\complement_{\varepsilon,\kappa}}} = \varepsilon\kappa \mathbf 1_{m- m_b}\}.
\end{equation*}
where the vector comparison $\succeq$ has to be understood elementwise.
Now, we are ready to define the \emph{screened dual of Sinkhorn divergence} as 
\begin{align}
\label{screen-sinkhorn_second_def}
\min_{u \in \mathcal{U}_{\text{sc}}, v \in \mathcal{V}_{\text{sc}}}\{\Psi_{\varepsilon, \kappa}(u,v)\}
\end{align}
where 
\begin{align*} 
\Psi_{\varepsilon,\kappa}(u, v) &= (e^{u_{I_{\varepsilon,\kappa}}})^\top K_{(I_{\varepsilon,\kappa}, J_{\varepsilon,\kappa})} e^{v_{J_{\varepsilon,\kappa}}} + 
\varepsilon \kappa (e^{u_{I_{\varepsilon,\kappa}}})^\top K_{(I_{\varepsilon,\kappa}, J^\complement_{\varepsilon,\kappa})}\mathbf 1_{m_b} + \varepsilon \kappa^{-1} \mathbf 1_{n_b}^\top K_{(I^\complement_{\varepsilon,\kappa}, J_{\varepsilon,\kappa})}e^{v_{J_{\varepsilon,\kappa}}}\\
&\qquad - \kappa \mu_{I_{\varepsilon,\kappa}}^\top u_{I_{\varepsilon,\kappa}} - \kappa^{-1} \nu_{J_{\varepsilon,\kappa}}^\top v_{J_{\varepsilon,\kappa}} + \Xi
\end{align*}
with $\Xi = \varepsilon^2 \sum_{i \in I^\complement_{\varepsilon,\kappa}, j \in J^\complement_{\varepsilon,\kappa}} K_{ij} -\kappa \log(\varepsilon\kappa^{-1})\sum_{i \in I^\complement_{\varepsilon,\kappa}}\mu_i - \kappa^{-1} \log(\varepsilon\kappa)\sum_{j\in J^\complement_{\varepsilon,\kappa}} \nu_j$.

Pseudocode of our proposed algorithm is given in Algorithm~\ref{screenkhorn}.
Screenkhorn consists of two steps: the first one is an initialization where we calculate the active sets $I_{\varepsilon,\kappa}$, $J_{\varepsilon,\kappa}$. 
The second is a constrained L-BFGS solver~\cite{zhu1997-LBFGS} for the stacked variable $\theta=(u_{I_{\varepsilon,\kappa}},v_{J_{\varepsilon,\kappa}}).$ 
% It is worth to note that the couple variables $(u,v)$ to be optimized in Screenkhorn belongs to $\R^{n_b\times m_b}$. 
It is worth to note that Screenkhorn uses only the restricted parts $K_{(I_{\varepsilon,\kappa}, J_{\varepsilon,\kappa})},$ $K_{(I_{\varepsilon,\kappa}, J^\complement_{\varepsilon,\kappa})},$ and $K_{(I^\complement_{\varepsilon,\kappa}, J_{\varepsilon,\kappa})}$ of the Gibbs kernel $K$, in contrast to Sinkhorn algorithm which performs alternating updates of all rows and columns of $K.$

The following lemma expresses upper and lower bounds to be respected in Screenkhorn.
\begin{lemma}
\label{lemma_bounds_of_usc_and_vsc}
Let $(u^{\text{sc}}, v^{\text{sc}})$ be an optimal solution of problem~\eqref{screen-sinkhorn_second_def}. Then,
one has
\begin{equation}
\label{bound_on_u}
\frac \varepsilon\kappa \vee \frac{\min_{i \in I_{\varepsilon,\kappa}}\mu_i}{\varepsilon (m- m_b) + \varepsilon \vee \frac{\max_{j\in J_{\varepsilon,\kappa}} \nu_j}{n\varepsilon\kappa\min_{i,j}K_{ij}} m_b} \leq e^{u^{\text{sc}}_i} \leq \frac \varepsilon\kappa\vee \frac{\max_{i \in I_{\varepsilon,\kappa}} \mu_i}{m\varepsilon\min_{i,j}K_{ij}},
\end{equation}
and
\begin{equation}
\label{bound_on_v}
\varepsilon\kappa \vee \frac{\min_{j \in J_{\varepsilon,\kappa}}\nu_j}{\varepsilon(n- n_b) + \varepsilon \vee \frac{\kappa\max_{i\in I_{\varepsilon,\kappa}} \mu_i}{m\varepsilon\min_{i,j}K_{ij}} n_b} \leq e^{v^{\text{sc}}_j} \leq \varepsilon\kappa \vee \frac{\max_{j \in J_{\varepsilon,\kappa}} \nu_j}{n\varepsilon\min_{i,j}K_{ij}}
\end{equation}
for all $i\in I_{\varepsilon,\kappa}$ and $j\in J_{\varepsilon,\kappa}$.
\end{lemma}

\LinesNotNumbered
\begin{algorithm}[htbp]
\SetNlSty{textbf}{}{.}
\DontPrintSemicolon
\caption{Screenkhorn$(C,\eta,\mu,\nu,n_b,m_b)$}
\label{screenkhorn}
% \textbf{input: }{$C$, $\eta$, $\mu \in \Sigma_n$, $\nu \in \Sigma_m$, $n_b$ and $m_b$;}\\

\textbf{step 1:} \textcolor{black}{Initialization}\\

% \nl   $K \gets \exp(-C/\eta);$\\
\nl   $\xi \gets \mu \oslash r(K);$ \\
\nl   $\zeta \gets \nu \oslash c(K);$\\
\nl   $\xi \gets \texttt{sort}(\xi);$ //(decreasing order)\\
\nl   $\zeta \gets \texttt{sort}(\zeta);$ //(decreasing order)\\
\nl   $\varepsilon \gets (\xi_{n_b}\zeta_{m_b})^{1/4}, \text{  } \kappa \gets \sqrt{{\zeta_{m_b}}/{\xi_{n_b}}}$;\\
\nl   $I_{\varepsilon,\kappa} \gets \{i=1, \ldots, n: \mu_i \geq {\varepsilon^2} \kappa^{-1} r_i(K)\};$\\
\nl   $J_{\varepsilon,\kappa} \gets \{j=1, \ldots, m: \nu_j \geq \varepsilon^2\kappa c_j(K)\};$\\ 

% \textbf{step 2:} \textcolor{black}{Bounds for L-BFGS}

\nl   $K_{\min} \gets \min_{i \in I_{\varepsilon,\kappa}, j \in J_{\varepsilon,\kappa}}K_{ij};$ \\
\nl   $\underline{\mu} \gets \min_{i \in I_{\varepsilon,\kappa}} \mu_i, \bar{\mu} \gets \max_{i \in I_{\varepsilon,\kappa}} \mu_i$; \\
\nl   $\underline{\nu} \gets \min_{j \in J_{\varepsilon,\kappa}} \mu_i, \bar{\nu} \gets \max_{j \in J_{\varepsilon,\kappa}} \mu_i$; \\
\nl   $\underline{u} \gets \log\big(\frac \varepsilon\kappa \vee \frac{\underline{\mu}}{\varepsilon (m-m_b) + \varepsilon \vee \frac{\bar{\nu}}{n\varepsilon\kappa K_{\min}} m_b}\big), \bar{u} \gets  \log\big(\frac \varepsilon\kappa\vee \frac{\bar{\mu}}{m\varepsilon K_{\min}}\big);$\\
\nl   $\underline{v} \gets \log\big(\varepsilon\kappa \vee \frac{\underline{\nu}}{\varepsilon(n-n_b) + \varepsilon \vee \frac{\kappa\bar{\mu}}{m\varepsilon K_{\min}} n_b}\big), \bar{v} \gets \log\big(\varepsilon\kappa \vee \frac{\bar{\nu}}{n\varepsilon K_{\min}}\big);$\\
\nl   $ \bar{\theta} \gets \text{stack}(\bar{u}\mathbf 1_{n_b}, \bar{v}\mathbf 1_{m_b});$\\
\nl   $ \underline{\theta} \gets \text{stack}(\underline{u}\mathbf 1_{n_b}, \underline{v}\mathbf 1_{m_b}) ;$\\ %\in \R^{n_b\times m_b}

\textbf{step 2:} \textcolor{black}{L-BFGS}\\

\nl  $u^{(0)} \gets \log(\varepsilon\kappa^{-1}) \mathbf 1_{n_b} ;$\\
\nl  $v^{(0)} \gets \log(\varepsilon\kappa) \mathbf 1_{m_b} ;$\\
\nl  $\theta^{(0)} \gets \text{stack}[u^{(0)}, v^{(0)}];$\\
\nl   $\theta \gets \text{L-BFGS}(\theta^{(0)}, \underline{\theta}, \bar{\theta});$\\
\nl   $\theta_u \gets (\theta_1, \ldots, \theta_{n_b})^\top, \theta_v \gets(\theta_{n_b+1}, \ldots, \theta_{n_b+m_b})^\top;$\\

% \textbf{step 4:} \textcolor{black}{Output}\\

\nl   {$u^{sc}_i \gets (\theta_u)_i$ if $i \in I_{\varepsilon,\kappa}$ and $u_i \gets \log(\varepsilon\kappa^{-1})$ if $i \in I^\complement_{\varepsilon,\kappa};$}\\
\nl   {$v^{sc}_j \gets (\theta_v)_j$ if $j \in J_{\varepsilon,\kappa}$ and $v_j \gets \log(\varepsilon\kappa)$ if $j \in J^\complement_{\varepsilon,\kappa};$}\\
\nl   \Return{$B(u^{\text{sc}},v^{\text{sc}})$.}
\end{algorithm}


\section{Analysis of marginal violations}
\label{sec:error_analysis}

This section is devoted to study the marginal violations of Screenkhorn. Towards this end, let us define the screened marginals $\mu^{\text{sc}} = B(u^{\text{sc}}, v^{\text{sc}}) \mathbf 1_m$ and $\nu^{\text{sc}} = B(u^{\text{sc}}, v^{\text{sc}})^\top \mathbf 1_n.$ 
Lemma~\ref{lemma_bounds_on_marginals} expresses an upper bound with respect to $\ell_1$-norm of $\mu^{\text{sc}}$and $\nu^{\text{sc}}$. % while Proposition~\ref{proposition_error_in_marginals} gives also an upper bound of the marginal errors.


\begin{lemma}
\label{lemma_bounds_on_marginals}
Let $(u^{\text{sc}}, v^{\text{sc}})$ be an optimal solution of problem~\eqref{screen-sinkhorn_second_def}.
Then one has 
\begin{equation*}
\norm{\mu^{\text{sc}}}_1 \leq \kappa \norm{\mu_{I_{\varepsilon,\kappa}}^{\text{}}}_1 + (n-n_b) \Big(\frac{m_b\max_{j \in J_{\varepsilon,\kappa}} \nu_j}{n\kappa\min_{i,j}K_{ij}} + (m-m_b)\varepsilon^2 \Big)
\end{equation*}
and 
\begin{equation*}
\norm{\nu^{\text{sc}}}_1 \leq \kappa^{-1} \norm{\nu_{J_{\varepsilon,\kappa}}}_1 + (m-m_b)\Big(\frac{n_b\kappa \max_{i\in I_{\varepsilon,\kappa}}\mu_i}{m\min_{i,j}K_{ij}} + (n-n_b)\varepsilon^2\Big).
\end{equation*}
\end{lemma}
The following Proposition gives also an upper bound of the marginal errors.
\begin{proposition}
\label{proposition_error_in_marginals}
One has 
\begin{align*} %{\norm{{\mu}^{\text{sc}}}_1} 
{\norm{{\mu} -{\mu}^{\text{sc}}}^2_1}&\leq 7n_b(\kappa-\log(\kappa)-1)\max_{i} \mu_i + 7 (n-n_b)\bigg(\frac{m_b\max_{j}\nu_j}{n\kappa\min_{i,j} K_{ij}} + (m- m_b) \varepsilon^2 - \min_{i}\mu_i\\
&+ \max_{i} \mu_i\log\Big(\frac{\kappa(n-n_b+ 1)\max_{i} \mu_i}{m_b\min_{i,j}K_{ij}\min_{j \in J_{\varepsilon,\kappa}}\nu_j} + \frac{\kappa^2\max_{i} \mu_i}{mm_b\varepsilon^2(\min_{i,j}K_{ij})^2\min_{j \in J_{\varepsilon,\kappa}}\nu_j}\Big)% \bigg\}^{1/2}
\end{align*}
and 
\begin{align*} %{\norm{{\nu}^{\text{sc}}}_1}
{\norm{{\nu} -{\nu}^{\text{sc}}}^2_1} &\leq 7m_b(\kappa-\log(\kappa)-1)\max_{j} \nu_j + 7(m-m_b)\bigg(\frac{n_b\kappa\max_{i}\mu_i}{n\min_{i,j} K_{ij}} + (n- n_b) \varepsilon^2 - \min_{j}\nu_j\\
&+ \max_{j} \nu_j\log\Big(\frac{\kappa(m-m_b+ 1)\max_{j} \nu_j}{n_b\min_{i,j}K_{ij}\min_{i \in I_{\varepsilon,\kappa}}\mu_i} + \frac{\kappa^2\max_{j} \nu_j}{nn_b\varepsilon^2(\min_{i,j}K_{ij})^2\min_{i \in I_{\varepsilon,\kappa}}\mu_i}\Big).% \bigg\}^{1/2}
\end{align*}


\end{proposition}

\section{Numerical experiments}
\label{sec:numerical_experiments}

% \subsubsection*{Acknowledgments}

% Use unnumbered third level headings for the acknowledgments. All acknowledgments go at the end of the paper. Do not include acknowledgments in the anonymized submission, only in the final paper.


\small
\bibliography{biblio}
\bibliographystyle{plain}

\end{document}
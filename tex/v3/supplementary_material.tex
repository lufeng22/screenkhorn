 \documentclass{article}

% if you need to pass options to natbib, use, e.g.:
% \PassOptionsToPackage{numbers, compress}{natbib}
% before loading neurips_2019

% ready for submission
\usepackage{neurips_2019}

% to compile a preprint version, e.g., for submission to arXiv, add add the
% [preprint] option:
% \usepackage[preprint]{neurips_2019}

% to compile a camera-ready version, add the [final] option, e.g.:
% \usepackage[]{neurips_2019}

% to avoid loading the natbib package, add option nonatbib:
% \usepackage[nonatbib]{neurips_2019}

\usepackage[utf8]{inputenc} % allow utf-8 input
\usepackage[T1]{fontenc}    % use 8-bit T1 fonts
\usepackage{hyperref}       % hyperlinks
\usepackage{url}            % simple URL typesetting
\usepackage{booktabs}       % professional-quality tables
\usepackage{amsfonts}       % blackboard math symbols
\usepackage{nicefrac}       % compact symbols for 1/2, etc.
\usepackage{microtype}      % microtypography

%--------------------------------------------------------------------------------------------------------------------------------% 

\usepackage{xr}
\externaldocument{main}

\RequirePackage{mathrsfs, amsthm, amsmath, amsfonts, amssymb, mathtools}%
\usepackage[titletoc,title]{appendix}%
\usepackage{dsfont}%
\usepackage{caption} % 
\usepackage{subcaption}
\usepackage{float}%
\usepackage{graphicx, color}% 
\usepackage{bm}%bold equations

%% formatting appendices
\usepackage{etoolbox}
\patchcmd{\appendices}{\quad}{. }{}{}

%% tables
\usepackage{pgfplots}%
\usepackage{booktabs,multirow,array,multicol}
\newcommand{\otoprule}{\midrule[\heavyrulewidth]}

%% algorithms
\usepackage[linesnumbered,ruled,vlined]{algorithm2e}

\usepackage{accents}
\usepackage{datenumber}
\usepackage{wrapfig}

%% sections
\newtheorem{theorem}{Theorem}
\newtheorem{corollary}{Corollary}
\newtheorem{lemma}{Lemma}
\newtheorem{fact}{Fact}
\newtheorem{proposition}{Proposition}
\newtheorem{definition}{Definition}%[section]
\newtheorem*{notation}{Notation}%[section]
\newtheorem{discus}{Discussion}%[section]
\newtheorem{remark}{Remark}%[section]
\newtheorem{example}{Example}%[section]
\newtheorem{exs}{Examples}%[section]
\newtheorem{ca}{Cas}
\newtheorem{remarks}{Remarks}%[section]
\newtheorem{assumption}{Assumption}{\bf}{\rm}%

% math macros
\newcommand{\bigO}{\mathcal{O}}
\newcommand{\inr}[1]{\langle #1 \rangle}
\newcommand{\ind}[1]{{\mathds{1}}_{{#1}}}%
\newcommand{\norm}[1]{\|#1\|}
\newcommand{\R}{{\mathbb{R}}}
\newcommand{\E}{\mathds{E}}
\newcommand{\bcdot}{\raisebox{-0.80ex}{\scalebox{1.8}{$\cdot$}}}
\newcommand{\varsig}{\raisebox{-0.15ex}{\scalebox{1.30}{$\varsigma$}}}
\newcommand*{\dt}[1]{\accentset{\mbox{\large\bfseries .}}{#1}}

%% math operators
\DeclareMathOperator*{\argmin}{\arg\!\min}
\DeclareMathOperator*{\argmax}{\arg\!\max}
%--------------------------------------------------------------------------------------------------------------------------------%

\title{Supplementary material for ``Screening Sinkhorn Algorithm for regularized Optimal Transport''}

\begin{document}

\author{%
Mokhtar Z. Alaya \\
LITIS EA4108\\
University of Rouen\\
\texttt{mokhtarzahdi.alaya@gmail.com} 
\And
Maxime Bérar\\
LITIS EA4108\\
University of Rouen\\
\texttt{maxime.berar@univ-rouen.fr} \\
\And
Gilles Gasso \\
LITIS EA4108\\
INSA, University of Rouen\\
\texttt{gilles.gasso@insa-rouen.fr} 
\And
Alain Rakotomamonjy\\
LITIS EA4108 \\
University of Rouen\\
and Criteo AI Labs, Criteo Paris \\
\texttt{alain.rakoto@insa-rouen.fr} \\
}
\maketitle
%
%\section{Proof of Lemma~\ref{lemma_actives_sets}}
%
%Since the objective function $\Psi_{\kappa}$ is convex with respect to $(u,v)$, the set of optima of problem~\eqref{screen-sinkhorn} is non empty.
%Introducing two dual variables $\lambda \in \R^n_{+}$ and $\beta \in \R^m_{+}$ for each constraint, the Lagrangian of problem~\eqref{screen-sinkhorn} reads as 
%\begin{equation*}
%  \mathscr{L}(u,v, \lambda, \beta) = \frac \varepsilon\kappa\inr{\lambda, \mathbf{1}_n} + \varepsilon\kappa\inr{\beta, \mathbf{1}_m} + \mathbf{1}_n^\top B(u,v) \mathbf{1}_m - \inr{\kappa u, \mu} - \inr{\frac v\kappa, \nu} -\inr{\lambda,e^{u}} - \inr{\beta,e^{v}}
%\end{equation*}
%First order conditions then yield that the Lagrangian multiplicators solutions $\lambda^{*}$ and $\beta^{*}$ satisfy 
%\begin{align*}
%  &\nabla_u\mathscr{L}(u^{*},v^{*}, \lambda^{*}, \beta^{*})=  e^{u^{*}} \odot(Ke^{v^{*}} - \lambda^{*}) - \kappa\mu = \mathbf 0_n,\\
%  & \text{ and } \nabla_v\mathscr{L}(u^{*},v^{*}, \lambda^{*}, \beta^{*})=  e^{v^{*}} \odot(K^\top e^{u^{*}} - \beta) - \frac \nu\kappa = \mathbf 0_m
%\end{align*}
%which leads to 
%\begin{align*}
%  &\lambda^{*} = K e^{v^{*}} - \kappa\mu \oslash e^{u^{*}} \text{ and }
%  \beta^{*} = K^\top e^{u^{*}} - \nu \oslash \kappa e^{v^{*}}
%\end{align*}
%
%For all $i=1, \ldots, n$ we have that $e^{u^{*}_i} \geq \varepsilon\kappa^{-1}$. Further, the condition on the dual variable $\lambda^{*}_i > 0$  ensures that $e^{u^{*}_i} = \varepsilon\kappa^{-1}$ and hence $i \in I^\complement_{\varepsilon,\kappa}$. We have that $\lambda^{*}_i > 0$ is equivalent to $e^{u^{*}_i}r_i(K) e^{v^{*}_j} >  \kappa{\mu_i}$ which  is satisfied when $\varepsilon^2r_i(K) >  \kappa{\mu_i}.$  
%In a symmetric way we can prove the same statement for $e^{v^{*}_j}$.
%
%\section{Proof of Proposition~\ref{prop:bounds_of_usc_and_vsc}}
%
%We prove only the first statement~\eqref{bound_on_u} and similarly we can prove the second one~\eqref{bound_on_v}.
%For all $i\in I_{\varepsilon,\kappa}$, we have $e^{u^{\text{sc}}_i} > \frac \varepsilon\kappa$ or $e^{u^{\text{sc}}_i} = \frac \varepsilon\kappa$. In one hand, if $e^{u^{\text{sc}}_i} > \frac \varepsilon\kappa$ then according to the optimality conditions $\lambda^{\text{sc}}_i = 0,$ which implies $e^{u^{\text{sc}}_i} \sum_{j=1}^m K_{ij} e^{v^{\text{sc}}_j} = \kappa\mu_i$.
%In another hand, we have 
%\begin{align*}
%e^{u^{\text{sc}}_i} \min_{i,j}K_{ij} \sum_{j=1}^m e^{v^{\text{sc}}_j} \leq e^{u^{\text{sc}}_i} \sum_{j=1}^m K_{ij} e^{v^{\text{sc}}_j} = \kappa\mu_i.
%\end{align*}
%We further observe that $\sum_{j=1}^m e^{v^{\text{sc}}_j} = \sum_{j \in J_{\varepsilon,\kappa}} e^{v^{\text{sc}}_j} + \sum_{j \in J^\complement_{\varepsilon,\kappa}} e^{v^{\text{sc}}_j} \geq \varepsilon\kappa |J_{\varepsilon,\kappa}| + \varepsilon\kappa |J^\complement_{\varepsilon,\kappa}|=\varepsilon\kappa m.$ Then
%\begin{equation*}
%\max_{i\in I_{\varepsilon,\kappa}} e^{u^{\text{sc}}_i} \leq \frac \varepsilon\kappa \vee \frac{\max_{i\in I_{\varepsilon,\kappa}}\mu_i}{m\varepsilon K_{\min}} \leq \frac \varepsilon\kappa \vee \frac{\max_{i\in I_{\varepsilon,\kappa}}\mu_i}{m\varepsilon K_{\min}}.
%\end{equation*}
%Analogously, one can obtain for all $j\in J_{\varepsilon,\kappa}$
%\begin{equation}
%\label{upper_bound_v_potential}
%\max_{j\in J_{\varepsilon,\kappa}}e^{v^{\text{sc}}_j} \leq \varepsilon\kappa \vee \frac{\max_{j \in J_{\varepsilon,\kappa}} \nu_j}{n\varepsilon K_{\min}} \leq \varepsilon\kappa \vee \frac{\max_{j \in J_{\varepsilon,\kappa}} \nu_j}{n\varepsilon K_{\min}} .
%\end{equation}
%
%Now, since $K_{ij} \leq 1$, we have 
%\begin{align*}
%e^{u^{\text{sc}}_i} \sum_{j=1}^m e^{v^{\text{sc}}_j} \geq e^{u^{\text{sc}}_i} \sum_{j=1}^m K_{ij}e^{v^{\text{sc}}_j} = \kappa\mu_i.
%\end{align*}
%Using~\eqref{upper_bound_v_potential}, we get 
%\begin{align*}
%\sum_{j=1}^m e^{v^{\text{sc}}_j} &= \sum_{j \in J_{\varepsilon,\kappa}} e^{v^{\text{sc}}_j} + \sum_{j \in J^\complement_{\varepsilon,\kappa}} e^{v^{\text{sc}}_j}
%\leq \varepsilon\kappa |J^\complement_{\varepsilon,\kappa}| + \varepsilon\kappa \vee \frac{\max_{j\in J_{\varepsilon,\kappa}} \nu_j}{n\varepsilon K_{\min}} |J_{\varepsilon,\kappa}|.
%\end{align*}
%Therefore,
%\begin{align*}
%\min_{i \in I_{\varepsilon,\kappa}} e^{u^{\text{sc}}_i}  \geq \frac \varepsilon\kappa \vee \frac{\kappa\min_{I_{\varepsilon,\kappa}}\mu_i}{\varepsilon\kappa (m-m_b) + \varepsilon\kappa \vee \frac{\max_{j\in J_{\varepsilon,\kappa}} \nu_j}{n\varepsilon K_{\min}} m_b}.
%\end{align*}
%
%\section{Proof of Lemma~\ref{lemma_bounds_on_marginals}}
%
%
%The optimality condition for $({u}^{\text{sc}}, {v}^{\text{sc}})$ entails 
%\begin{align}
%\label{i-th-marginal-mu} 
%{\mu}^{\text{sc}}_i  &= 
%\begin{cases}
%e^{u^{\text{sc}}_i} \sum_{j=1}^m K_{ij} e^{v^{\text{sc}}_j}, \text{ if  }i \in I_{\varepsilon,\kappa},\\
%\frac \varepsilon\kappa\sum_{j=1}^m K_{ij} e^{v^{\text{sc}}_j}, \text{ if  }i \in I^\complement_{\varepsilon,\kappa}
%\end{cases}
%=\begin{cases}
%\kappa \mu_i, \text{ if  }i \in I_{\varepsilon,\kappa},\\
%\frac \varepsilon\kappa\sum_{j=1}^m K_{ij} e^{v^{\text{sc}}_j}, \text{ if  }i \in I^\complement_{\varepsilon,\kappa},
%\end{cases}
%\end{align}
%and 
%\begin{align}
%\label{i-th-marginal-nu} 
%{\nu}^{\text{sc}}_j  &= 
%\begin{cases}
%e^{v^{\text{sc}}_j} \sum_{i=1}^n K_{ij} e^{u^{\text{sc}}_i}, \text{ if  }j \in J_{\varepsilon,\kappa},\\
%\varepsilon\kappa\sum_{i=1}^n K_{ij} e^{u^{\text{sc}}_i}, \text{ if  }j \in J^\complement_{\varepsilon,\kappa}
%\end{cases}
%=\begin{cases}
%\frac{\nu_j}{\kappa}, \text{ if  }j \in J_{\varepsilon,\kappa},\\
%\varepsilon\kappa\sum_{i=1}^n K_{ij} e^{u^{\text{sc}}_i}, \text{ if  }j \in J^\complement_{\varepsilon,\kappa}.
%\end{cases}
%\end{align}
%
%Using inequality~\eqref{bound_on_v}, we obtain 
%\begin{align*}
%\norm{\mu^{\text{sc}}}_1 &= \sum_{i \in I_{\varepsilon,\kappa}} \mu^{\text{sc}}_i +  \sum_{i \in I^\complement_{\varepsilon,\kappa}}\mu^{\text{sc}}_i\\
%& \overset{\eqref{i-th-marginal-mu}}{=} \kappa \norm{\mu_{I_{\varepsilon,\kappa}}^{\text{sc}}}_1 + \frac \varepsilon\kappa \sum_{i \in I^\complement}\Big( \sum_{j \in J_{\varepsilon,\kappa}} K_{ij} e^{v^{\text{sc}}_j} + \varepsilon\kappa \sum_{j\in J^\complement_{\varepsilon,\kappa}}K_{ij}\Big)\\
%& \overset{\eqref{bound_on_v}}{\leq} \kappa \norm{\mu_{I_{\varepsilon,\kappa}}^{\text{sc}}}_1 + (n-n_b) \Big(\frac{m_b\max_{j \in J_{\varepsilon,\kappa}} \nu_j}{n\kappa K_{\min}} + (m-m_b)\varepsilon^2 \Big).
%\end{align*}
%Again by left-hand-side of inequaltiy~\eqref{bound_on_v}, we arrive at 
%\begin{align*}
%\norm{\mu^{\text{sc}}}_1 %&= \sum_{i \in I_{\varepsilon,\kappa}} \mu^{\text{sc}}_i +  \sum_{i \in I^\complement_{\varepsilon,\kappa}}\mu^{\text{sc}}_i\\
%%& \overset{\eqref{i-th-marginal-mu}}{=} \kappa \norm{\mu_{I_{\varepsilon,\kappa}}^{\text{sc}}}_1 + \frac \varepsilon\kappa \sum_{i \in I^\complement}\Big( \sum_{j \in J_{\varepsilon,\kappa}} K_{ij} e^{v^{\text{sc}}_j} + \varepsilon\kappa \sum_{j\in J^\complement_{\varepsilon,\kappa}}K_{ij}\Big)\\
%& \overset{}{\geq} \kappa \norm{\mu_{I_{\varepsilon,\kappa}}^{\text{sc}}}_1 + (n -n_b) \Big(\frac{mm_b\varepsilon^2 K_{\min}^2 \min_{j\in J_{\varepsilon, \kappa}} \nu_j}{ (n-n_b)m\kappa \varepsilon^2 K_{\min} + n_b\kappa^2 \max_{i\in I_{\varepsilon,\kappa}}\mu_i }+ (m-m_b)\varepsilon^2K_{\min}\Big),
%\end{align*}
%which gives the claimed result.
%Similarly, we can prove the same statement for $\norm{\nu^{\text{sc}}}_1$.
%
%\section{Proof of Proposition~\ref{proposition_error_in_marginals}}
%
%We define the distance function $\varrho: \R_+ \times \R_+ \mapsto [0, \infty]$ by $\varrho(a,b) = b - a + a \log(\frac ab).$
%While $\varrho$ is not a metric, it is easy to see that $\varrho$ is not nonnegative and satisfies $\varrho(a,b) =0$ iff $a=b$.
%The violations are computed through the following function: 
%\begin{equation*}
%	d_{\varrho}(\gamma,\beta) = \sum_{i=1}^n \varrho(\gamma_i,\beta_i), \text{ for } \gamma, \beta \in \R^n_+.
%\end{equation*}
%Note that if $\gamma,\beta$ are two vectors of positive entries, $d_{\varrho}(\gamma,\beta)$ will return some measurement on how far they are from each other. The next Lemma is from~\cite{khalilabid2018} (see Lemma 7 herein).
%\begin{lemma}
%\label{lem:pinsker}
%For any $\gamma, \beta \in \R^n_+$, the following generalized Pinsker inequality holds 
%\begin{align*}
%\norm{\gamma - \beta}_1 \leq \sqrt{7 (\norm{\gamma}_1\wedge \norm{\beta}_1)d_{\varrho}(\gamma,\beta)}
%\end{align*}
%\end{lemma}
%By~\eqref{i-th-marginal-mu}, we have
%\begin{align*}
%d_\varrho({\mu} ,{\mu}^{\text{sc}}) &= \sum_{i=1}^n  {\mu}^{\text{sc}}_i - {\mu}_i + {\mu}_i  \log\Big(\frac{{\mu}_i}{{\mu}^{\text{sc}}_i }\Big)\\
%&= \sum_{i\in I_{\varepsilon,\kappa}} (\kappa-1)\mu_i - \mu_i\log(\kappa) + \sum_{i\in I^\complement_{\varepsilon,\kappa}}\frac \varepsilon\kappa\sum_{j=1}^m K_{ij} e^{v^{\text{sc}}_j} - \mu_i + \mu_i \log\Big(\frac{\mu_i}{\frac \varepsilon\kappa\sum_{j=1}^m K_{ij} e^{v^{\text{sc}}_j}}\Big)\\
%&= \sum_{i\in I_{\varepsilon,\kappa}} (\kappa-\log(\kappa)-1)\mu_i  + \sum_{i\in I^\complement_{\varepsilon,\kappa}}\frac \varepsilon\kappa\sum_{j=1}^m K_{ij} e^{v^{\text{sc}}_j} - \mu_i + \mu_i \log\Big(\frac{\mu_i}{\frac \varepsilon\kappa\sum_{j=1}^m K_{ij} e^{v^{\text{sc}}_j}}\Big).
%% &\leq  \sum_{i\in I^\complement_{\varepsilon,\kappa}}\frac \varepsilon\kappa\sum_{j=1}^m K_{ij} e^{v^{\text{sc}}_j} - \mu_i + \mu_i \log\Big(\frac{\mu_i}{\frac \varepsilon\kappa\sum_{j=1}^m K_{ij} e^{v^{\text{sc}}_j}}\Big)
%\end{align*}
%Now by~\eqref{bound_on_v}, we have in one hand 
%\begin{align*}
%\sum_{i\in I^\complement_{\varepsilon,\kappa}}\frac \varepsilon\kappa\sum_{j=1}^m K_{ij} e^{v^{\text{sc}}_j}&= \sum_{i\in I^\complement_{\varepsilon,\kappa}}\frac \varepsilon\kappa \Big(\sum_{j\in J_{\varepsilon,\kappa}}K_{ij} e^{v^{\text{sc}}_j} + \varepsilon \kappa\sum_{j\in J^\complement_{\varepsilon,\kappa}}K_{ij}\Big)\\
%&\leq \sum_{i\in I^\complement_{\varepsilon,\kappa}}\frac \varepsilon\kappa \Big(m_b \max_{i,j}K_{ij}\frac{\max_{j \in J_{\varepsilon,\kappa}} \nu_j}{n\varepsilon K_{\min}} + (m - m_b)\varepsilon\kappa\max_{i,j}K_{ij}\Big) \\
%&\leq (n-n_b)\Big(\frac{m_b\max_{j} \nu_j}{n\kappa K_{\min}} + (m- m_b) \varepsilon^2\Big).
%\end{align*}
%On the other hand, we get
%\begin{align*}
%\frac \varepsilon\kappa\sum_{j=1}^m K_{ij} e^{v^{\text{sc}}_j}&=\frac \varepsilon\kappa \Big(\sum_{j\in J_{\varepsilon,\kappa}}K_{ij} e^{v^{\text{sc}}_j} + \varepsilon \kappa\sum_{j\in J^\complement_{\varepsilon,\kappa}}K_{ij}\Big)\\
%&\geq m_bK_{\min} \frac{m\varepsilon^2K_{\min}\min_{j \in J_{\varepsilon,\kappa}}\nu_j}{\kappa((n-n_b)m\varepsilon^2K_{\min} + m\varepsilon^2K_{\min} + n_b\kappa\max_{i\in I_{\varepsilon,\kappa}}\mu_i)}\\
%&\qquad +\varepsilon^2 (m- m_b) K_{\min}\\
%&\geq \frac{mm_b\varepsilon^2(K_{\min})^2\min_{j \in J_{\varepsilon,\kappa}}\nu_j}{\kappa((n-n_b)m\varepsilon^2K_{\min}+ m\varepsilon^2K_{\min} + n_b\kappa\max_{i\in I_{\varepsilon,\kappa}}\mu_i)}\\
%&\qquad +\varepsilon^2 (m- m_b) K_{\min}\\
%&\geq \frac{mm_b\varepsilon^2K_{\min}^2\min_{j \in J_{\varepsilon,\kappa}}\nu_j}{\kappa((n-n_b)m\varepsilon^2K_{\min}+ m\varepsilon^2K_{\min} + n_b\kappa\max_{i\in I_{\varepsilon,\kappa}}\mu_i)}.
%\end{align*}
%Then 
%\begin{align*}
%\frac{1}{\frac \varepsilon\kappa\sum_{j=1}^m K_{ij} e^{v^{\text{sc}}_j}} &\leq \frac{\kappa((n-n_b)m\varepsilon^2K_{\min}+ m\varepsilon^2K_{\min} + n_b\kappa\max_{i\in I_{\varepsilon,\kappa}}\mu_i)}{mm_b\varepsilon^2 K_{\min}^2\min_{j \in J_{\varepsilon,\kappa}}\nu_j}\\
%&\leq \frac{\kappa(n-n_b+ 1)}{m_bK_{\min}\min_{j \in J_{\varepsilon,\kappa}}\nu_j} + \frac{n_b\kappa^2\max_{i\in I_{\varepsilon,\kappa}}\mu_i}{mm_b\varepsilon^2K_{\min}^2\min_{j \in J_{\varepsilon,\kappa}}\nu_j}.
%\end{align*}
%It entails 
%\begin{align*}
%&\sum_{i\in I^\complement_{\varepsilon,\kappa}}\frac \varepsilon\kappa\sum_{j=1}^m K_{ij} e^{v^{\text{sc}}_j} - \mu_i + \mu_i \log\Big(\frac{\mu_i}{\frac \varepsilon\kappa\sum_{j=1}^m K_{ij} e^{v^{\text{sc}}_j}}\Big)\\
%&\leq (n-n_b)\bigg(\frac{m_b}{n\kappa\min_{i,j} K_{ij}} + (m- m_b) \varepsilon^2 - \min_{i}\mu_i\\
%&\qquad + \max_{i}\mu_i\log\Big(\frac{\kappa(n-n_b+ 1)\max_{i}\mu_i}{m_bK_{\min}\min_{j \in J_{\varepsilon,\kappa}}\nu_j} + \frac{n_b\kappa^2(\max_{i}\mu_i)^2}{mm_b\varepsilon^2 K_{\min}^2\min_{j \in J_{\varepsilon,\kappa}}\nu_j}\Big)
%\bigg).
%\end{align*}
%Therefore
%\begin{align*}
%d_\varrho({\mu},{\mu}^{\text{sc}}) &\leq n_b(\kappa-\log(\kappa)-1)\max_{i} \mu_i + (n-n_b)\bigg(\frac{m_b\max_{j}\nu_j}{n\kappa\min_{i,j} K_{ij}} + (m- m_b) \varepsilon^2 - \min_{i}\mu_i\\
%&\qquad + \max_{i} \mu_i\log\Big(\frac{\kappa(n-n_b+ 1)\max_{i} \mu_i}{m_bK_{\min}\min_{j \in J_{\varepsilon,\kappa}}\nu_j} + \frac{n_b\kappa^2(\max_{i} \mu_i)^2}{mm_b\varepsilon^2 K_{\min}^2\min_{j \in J_{\varepsilon,\kappa}}\nu_j}\Big).
%\end{align*}
%Finally, by Lemma~\ref{lem:pinsker} we obtain
%\begin{align*}
%\norm{{\mu} -{\mu}^{\text{sc}}}^2_1 \leq & n_b(\kappa-\log(\kappa)-1)\max_{i} \mu_i + 7(n-n_b)\bigg(\frac{m_b\max_{j}\nu_j}{n\kappa\min_{i,j} K_{ij}} + (m- m_b) \varepsilon^2 - \min_{i}\mu_i\\
%&+ \max_{i} \mu_i\log\Big(\frac{\kappa(n-n_b+ 1)\max_{i} \mu_i}{m_bK_{\min}\min_{j \in J_{\varepsilon,\kappa}}\nu_j} + \frac{n_b\kappa^2(\max_{i} \mu_i)^2}{mm_b\varepsilon^2K_{\min}^2\min_{j \in J_{\varepsilon,\kappa}}\nu_j}\Big).%\bigg\}^{1/2}
%\end{align*}
%Proof of the upper bound for $\norm{\nu - {\nu}^{\text{sc}}}^2_1$ follows the same lines as above.

\subsection{Proof of Lemma~\ref{lemma_actives_sets}}

Since the objective function $\Psi_{\kappa}$ is convex with respect to $(u,v)$, the set of optima of problem~\eqref{screen-sinkhorn} is non empty.
Introducing two dual variables $\lambda \in \R^n_{+}$ and $\beta \in \R^m_{+}$ for each constraint, the Lagrangian of problem~\eqref{screen-sinkhorn} reads as 
\begin{equation*}
  \mathscr{L}(u,v, \lambda, \beta) = \frac \varepsilon\kappa\inr{\lambda, \mathbf{1}_n} + \varepsilon\kappa\inr{\beta, \mathbf{1}_m} + \mathbf{1}_n^\top B(u,v) \mathbf{1}_m - \inr{\kappa u, \mu} - \inr{\frac v\kappa, \nu} -\inr{\lambda,e^{u}} - \inr{\beta,e^{v}}
\end{equation*}
First order conditions then yield that the Lagrangian multiplicators solutions $\lambda^{*}$ and $\beta^{*}$ satisfy 
\begin{align*}
  &\nabla_u\mathscr{L}(u^{*},v^{*}, \lambda^{*}, \beta^{*})=  e^{u^{*}} \odot(Ke^{v^{*}} - \lambda^{*}) - \kappa\mu = \mathbf 0_n,\\
  & \text{ and } \nabla_v\mathscr{L}(u^{*},v^{*}, \lambda^{*}, \beta^{*})=  e^{v^{*}} \odot(K^\top e^{u^{*}} - \beta) - \frac \nu\kappa = \mathbf 0_m
\end{align*}
which leads to 
\begin{align*}
  &\lambda^{*} = K e^{v^{*}} - \kappa\mu \oslash e^{u^{*}} \text{ and }
  \beta^{*} = K^\top e^{u^{*}} - \nu \oslash \kappa e^{v^{*}}
\end{align*}

For all $i=1, \ldots, n$ we have that $e^{u^{*}_i} \geq \varepsilon\kappa^{-1}$. Further, the condition on the dual variable $\lambda^{*}_i > 0$  ensures that $e^{u^{*}_i} = \varepsilon\kappa^{-1}$ and hence $i \in I^\complement_{\varepsilon,\kappa}$. We have that $\lambda^{*}_i > 0$ is equivalent to $e^{u^{*}_i}r_i(K) e^{v^{*}_j} >  \kappa{\mu_i}$ which  is satisfied when $\varepsilon^2r_i(K) >  \kappa{\mu_i}.$  
In a symmetric way we can prove the same statement for $e^{v^{*}_j}$.

\subsection{Proof of Proposition~\ref{prop:bounds_of_usc_and_vsc}}

We prove only the first statement~\eqref{bound_on_u} and similarly we can prove the second one~\eqref{bound_on_v}.
For all $i\in I_{\varepsilon,\kappa}$, we have $e^{u^{\text{sc}}_i} > \frac \varepsilon\kappa$ or $e^{u^{\text{sc}}_i} = \frac \varepsilon\kappa$. In one hand, if $e^{u^{\text{sc}}_i} > \frac \varepsilon\kappa$ then according to the optimality conditions $\lambda^{\text{sc}}_i = 0,$ which implies $e^{u^{\text{sc}}_i} \sum_{j=1}^m K_{ij} e^{v^{\text{sc}}_j} = \kappa\mu_i$.
In another hand, we have 
\begin{align*}
e^{u^{\text{sc}}_i} \min_{i,j}K_{ij} \sum_{j=1}^m e^{v^{\text{sc}}_j} \leq e^{u^{\text{sc}}_i} \sum_{j=1}^m K_{ij} e^{v^{\text{sc}}_j} = \kappa\mu_i.
\end{align*}
We further observe that $\sum_{j=1}^m e^{v^{\text{sc}}_j} = \sum_{j \in J_{\varepsilon,\kappa}} e^{v^{\text{sc}}_j} + \sum_{j \in J^\complement_{\varepsilon,\kappa}} e^{v^{\text{sc}}_j} \geq \varepsilon\kappa |J_{\varepsilon,\kappa}| + \varepsilon\kappa |J^\complement_{\varepsilon,\kappa}|=\varepsilon\kappa m.$ Then
\begin{equation*}
\max_{i\in I_{\varepsilon,\kappa}} e^{u^{\text{sc}}_i} \leq \frac \varepsilon\kappa \vee \frac{\max_{i\in I_{\varepsilon,\kappa}}\mu_i}{m\varepsilon K_{\min}} \leq \frac \varepsilon\kappa \vee \frac{\max_{i\in I_{\varepsilon,\kappa}}\mu_i}{m\varepsilon K_{\min}}.
\end{equation*}
Analogously, one can obtain for all $j\in J_{\varepsilon,\kappa}$
\begin{equation}
\label{upper_bound_v_potential}
\max_{j\in J_{\varepsilon,\kappa}}e^{v^{\text{sc}}_j} \leq \varepsilon\kappa \vee \frac{\max_{j \in J_{\varepsilon,\kappa}} \nu_j}{n\varepsilon K_{\min}} \leq \varepsilon\kappa \vee \frac{\max_{j \in J_{\varepsilon,\kappa}} \nu_j}{n\varepsilon K_{\min}} .
\end{equation}

Now, since $K_{ij} \leq 1$, we have 
\begin{align*}
e^{u^{\text{sc}}_i} \sum_{j=1}^m e^{v^{\text{sc}}_j} \geq e^{u^{\text{sc}}_i} \sum_{j=1}^m K_{ij}e^{v^{\text{sc}}_j} = \kappa\mu_i.
\end{align*}
Using~\eqref{upper_bound_v_potential}, we get 
\begin{align*}
\sum_{j=1}^m e^{v^{\text{sc}}_j} &= \sum_{j \in J_{\varepsilon,\kappa}} e^{v^{\text{sc}}_j} + \sum_{j \in J^\complement_{\varepsilon,\kappa}} e^{v^{\text{sc}}_j}
\leq \varepsilon\kappa |J^\complement_{\varepsilon,\kappa}| + \varepsilon\kappa \vee \frac{\max_{j\in J_{\varepsilon,\kappa}} \nu_j}{n\varepsilon K_{\min}} |J_{\varepsilon,\kappa}|.
\end{align*}
Therefore,
\begin{align*}
\min_{i \in I_{\varepsilon,\kappa}} e^{u^{\text{sc}}_i}  \geq \frac \varepsilon\kappa \vee \frac{\kappa\min_{I_{\varepsilon,\kappa}}\mu_i}{\varepsilon\kappa (m-m_b) + \varepsilon\kappa \vee \frac{\max_{j\in J_{\varepsilon,\kappa}} \nu_j}{n\varepsilon K_{\min}} m_b}.
\end{align*}

\subsection{Proof of Lemma~\ref{lemma_bounds_on_marginals}}


The optimality condition for $({u}^{\text{sc}}, {v}^{\text{sc}})$ entails 
\begin{align}
\label{i-th-marginal-mu} 
{\mu}^{\text{sc}}_i  &= 
\begin{cases}
e^{u^{\text{sc}}_i} \sum_{j=1}^m K_{ij} e^{v^{\text{sc}}_j}, \text{ if  }i \in I_{\varepsilon,\kappa},\\
\frac \varepsilon\kappa\sum_{j=1}^m K_{ij} e^{v^{\text{sc}}_j}, \text{ if  }i \in I^\complement_{\varepsilon,\kappa}
\end{cases}
=\begin{cases}
\kappa \mu_i, \text{ if  }i \in I_{\varepsilon,\kappa},\\
\frac \varepsilon\kappa\sum_{j=1}^m K_{ij} e^{v^{\text{sc}}_j}, \text{ if  }i \in I^\complement_{\varepsilon,\kappa},
\end{cases}
\end{align}
and 
\begin{align}
\label{i-th-marginal-nu} 
{\nu}^{\text{sc}}_j  &= 
\begin{cases}
e^{v^{\text{sc}}_j} \sum_{i=1}^n K_{ij} e^{u^{\text{sc}}_i}, \text{ if  }j \in J_{\varepsilon,\kappa},\\
\varepsilon\kappa\sum_{i=1}^n K_{ij} e^{u^{\text{sc}}_i}, \text{ if  }j \in J^\complement_{\varepsilon,\kappa}
\end{cases}
=\begin{cases}
\frac{\nu_j}{\kappa}, \text{ if  }j \in J_{\varepsilon,\kappa},\\
\varepsilon\kappa\sum_{i=1}^n K_{ij} e^{u^{\text{sc}}_i}, \text{ if  }j \in J^\complement_{\varepsilon,\kappa}.
\end{cases}
\end{align}

Using inequality~\eqref{bound_on_v}, we obtain 
\begin{align*}
\norm{\mu^{\text{sc}}}_1 &= \sum_{i \in I_{\varepsilon,\kappa}} \mu^{\text{sc}}_i +  \sum_{i \in I^\complement_{\varepsilon,\kappa}}\mu^{\text{sc}}_i\\
& \overset{\eqref{i-th-marginal-mu}}{=} \kappa \norm{\mu_{I_{\varepsilon,\kappa}}^{\text{sc}}}_1 + \frac \varepsilon\kappa \sum_{i \in I^\complement}\Big( \sum_{j \in J_{\varepsilon,\kappa}} K_{ij} e^{v^{\text{sc}}_j} + \varepsilon\kappa \sum_{j\in J^\complement_{\varepsilon,\kappa}}K_{ij}\Big)\\
& \overset{\eqref{bound_on_v}}{\leq} \kappa \norm{\mu_{I_{\varepsilon,\kappa}}^{\text{sc}}}_1 + (n-n_b) \Big(\frac{m_b\max_{j \in J_{\varepsilon,\kappa}} \nu_j}{n\kappa K_{\min}} + (m-m_b)\varepsilon^2 \Big).
\end{align*}
Again by left-hand-side of inequaltiy~\eqref{bound_on_v}, we arrive at 
\begin{align*}
\norm{\mu^{\text{sc}}}_1 %&= \sum_{i \in I_{\varepsilon,\kappa}} \mu^{\text{sc}}_i +  \sum_{i \in I^\complement_{\varepsilon,\kappa}}\mu^{\text{sc}}_i\\
%& \overset{\eqref{i-th-marginal-mu}}{=} \kappa \norm{\mu_{I_{\varepsilon,\kappa}}^{\text{sc}}}_1 + \frac \varepsilon\kappa \sum_{i \in I^\complement}\Big( \sum_{j \in J_{\varepsilon,\kappa}} K_{ij} e^{v^{\text{sc}}_j} + \varepsilon\kappa \sum_{j\in J^\complement_{\varepsilon,\kappa}}K_{ij}\Big)\\
& \overset{}{\geq} \kappa \norm{\mu_{I_{\varepsilon,\kappa}}^{\text{sc}}}_1 + (n -n_b) \Big(\frac{mm_b\varepsilon^2 K_{\min}^2 \min_{j\in J_{\varepsilon, \kappa}} \nu_j}{ (n-n_b)m\kappa \varepsilon^2 K_{\min} + n_b\kappa^2 \max_{i\in I_{\varepsilon,\kappa}}\mu_i }+ (m-m_b)\varepsilon^2K_{\min}\Big),
\end{align*}
which gives the claimed result.
Similarly, we can prove the same statement for $\norm{\nu^{\text{sc}}}_1$.

\subsection{Proof of Proposition~\ref{proposition_error_in_marginals}}

We define the distance function $\varrho: \R_+ \times \R_+ \mapsto [0, \infty]$ by $\varrho(a,b) = b - a + a \log(\frac ab).$
While $\varrho$ is not a metric, it is easy to see that $\varrho$ is not nonnegative and satisfies $\varrho(a,b) =0$ iff $a=b$.
The violations are computed through the following function: 
\begin{equation*}
	d_{\varrho}(\gamma,\beta) = \sum_{i=1}^n \varrho(\gamma_i,\beta_i), \text{ for } \gamma, \beta \in \R^n_+.
\end{equation*}
Note that if $\gamma,\beta$ are two vectors of positive entries, $d_{\varrho}(\gamma,\beta)$ will return some measurement on how far they are from each other. The next Lemma is from~\cite{khalilabid2018} (see Lemma 7 herein).
\begin{lemma}
\label{lem:pinsker}
For any $\gamma, \beta \in \R^n_+$, the following generalized Pinsker inequality holds 
\begin{align*}
\norm{\gamma - \beta}_1 \leq \sqrt{7 (\norm{\gamma}_1\wedge \norm{\beta}_1)d_{\varrho}(\gamma,\beta)}
\end{align*}
\end{lemma}
By~\eqref{i-th-marginal-mu}, we have
\begin{align*}
d_\varrho({\mu} ,{\mu}^{\text{sc}}) &= \sum_{i=1}^n  {\mu}^{\text{sc}}_i - {\mu}_i + {\mu}_i  \log\Big(\frac{{\mu}_i}{{\mu}^{\text{sc}}_i }\Big)\\
&= \sum_{i\in I_{\varepsilon,\kappa}} (\kappa-1)\mu_i - \mu_i\log(\kappa) + \sum_{i\in I^\complement_{\varepsilon,\kappa}}\frac \varepsilon\kappa\sum_{j=1}^m K_{ij} e^{v^{\text{sc}}_j} - \mu_i + \mu_i \log\Big(\frac{\mu_i}{\frac \varepsilon\kappa\sum_{j=1}^m K_{ij} e^{v^{\text{sc}}_j}}\Big)\\
&= \sum_{i\in I_{\varepsilon,\kappa}} (\kappa-\log(\kappa)-1)\mu_i  + \sum_{i\in I^\complement_{\varepsilon,\kappa}}\frac \varepsilon\kappa\sum_{j=1}^m K_{ij} e^{v^{\text{sc}}_j} - \mu_i + \mu_i \log\Big(\frac{\mu_i}{\frac \varepsilon\kappa\sum_{j=1}^m K_{ij} e^{v^{\text{sc}}_j}}\Big).
% &\leq  \sum_{i\in I^\complement_{\varepsilon,\kappa}}\frac \varepsilon\kappa\sum_{j=1}^m K_{ij} e^{v^{\text{sc}}_j} - \mu_i + \mu_i \log\Big(\frac{\mu_i}{\frac \varepsilon\kappa\sum_{j=1}^m K_{ij} e^{v^{\text{sc}}_j}}\Big)
\end{align*}
Now by~\eqref{bound_on_v}, we have in one hand 
\begin{align*}
\sum_{i\in I^\complement_{\varepsilon,\kappa}}\frac \varepsilon\kappa\sum_{j=1}^m K_{ij} e^{v^{\text{sc}}_j}&= \sum_{i\in I^\complement_{\varepsilon,\kappa}}\frac \varepsilon\kappa \Big(\sum_{j\in J_{\varepsilon,\kappa}}K_{ij} e^{v^{\text{sc}}_j} + \varepsilon \kappa\sum_{j\in J^\complement_{\varepsilon,\kappa}}K_{ij}\Big)\\
&\leq \sum_{i\in I^\complement_{\varepsilon,\kappa}}\frac \varepsilon\kappa \Big(m_b \max_{i,j}K_{ij}\frac{\max_{j \in J_{\varepsilon,\kappa}} \nu_j}{n\varepsilon K_{\min}} + (m - m_b)\varepsilon\kappa\max_{i,j}K_{ij}\Big) \\
&\leq (n-n_b)\Big(\frac{m_b\max_{j} \nu_j}{n\kappa K_{\min}} + (m- m_b) \varepsilon^2\Big).
\end{align*}
On the other hand, we get
\begin{align*}
\frac \varepsilon\kappa\sum_{j=1}^m K_{ij} e^{v^{\text{sc}}_j}&=\frac \varepsilon\kappa \Big(\sum_{j\in J_{\varepsilon,\kappa}}K_{ij} e^{v^{\text{sc}}_j} + \varepsilon \kappa\sum_{j\in J^\complement_{\varepsilon,\kappa}}K_{ij}\Big)\\
&\geq m_bK_{\min} \frac{m\varepsilon^2K_{\min}\min_{j \in J_{\varepsilon,\kappa}}\nu_j}{\kappa((n-n_b)m\varepsilon^2K_{\min} + m\varepsilon^2K_{\min} + n_b\kappa\max_{i\in I_{\varepsilon,\kappa}}\mu_i)}\\
&\qquad +\varepsilon^2 (m- m_b) K_{\min}\\
&\geq \frac{mm_b\varepsilon^2(K_{\min})^2\min_{j \in J_{\varepsilon,\kappa}}\nu_j}{\kappa((n-n_b)m\varepsilon^2K_{\min}+ m\varepsilon^2K_{\min} + n_b\kappa\max_{i\in I_{\varepsilon,\kappa}}\mu_i)}\\
&\qquad +\varepsilon^2 (m- m_b) K_{\min}\\
&\geq \frac{mm_b\varepsilon^2K_{\min}^2\min_{j \in J_{\varepsilon,\kappa}}\nu_j}{\kappa((n-n_b)m\varepsilon^2K_{\min}+ m\varepsilon^2K_{\min} + n_b\kappa\max_{i\in I_{\varepsilon,\kappa}}\mu_i)}.
\end{align*}
Then 
\begin{align*}
\frac{1}{\frac \varepsilon\kappa\sum_{j=1}^m K_{ij} e^{v^{\text{sc}}_j}} &\leq \frac{\kappa((n-n_b)m\varepsilon^2K_{\min}+ m\varepsilon^2K_{\min} + n_b\kappa\max_{i\in I_{\varepsilon,\kappa}}\mu_i)}{mm_b\varepsilon^2 K_{\min}^2\min_{j \in J_{\varepsilon,\kappa}}\nu_j}\\
&\leq \frac{\kappa(n-n_b+ 1)}{m_bK_{\min}\min_{j \in J_{\varepsilon,\kappa}}\nu_j} + \frac{n_b\kappa^2\max_{i\in I_{\varepsilon,\kappa}}\mu_i}{mm_b\varepsilon^2K_{\min}^2\min_{j \in J_{\varepsilon,\kappa}}\nu_j}.
\end{align*}
It entails 
\begin{align*}
&\sum_{i\in I^\complement_{\varepsilon,\kappa}}\frac \varepsilon\kappa\sum_{j=1}^m K_{ij} e^{v^{\text{sc}}_j} - \mu_i + \mu_i \log\Big(\frac{\mu_i}{\frac \varepsilon\kappa\sum_{j=1}^m K_{ij} e^{v^{\text{sc}}_j}}\Big)\\
&\leq (n-n_b)\bigg(\frac{m_b}{n\kappa\min_{i,j} K_{ij}} + (m- m_b) \varepsilon^2 - \min_{i}\mu_i\\
&\qquad + \max_{i}\mu_i\log\Big(\frac{\kappa(n-n_b+ 1)\max_{i}\mu_i}{m_bK_{\min}\min_{j \in J_{\varepsilon,\kappa}}\nu_j} + \frac{n_b\kappa^2(\max_{i}\mu_i)^2}{mm_b\varepsilon^2 K_{\min}^2\min_{j \in J_{\varepsilon,\kappa}}\nu_j}\Big)
\bigg).
\end{align*}
Therefore
\begin{align*}
d_\varrho({\mu},{\mu}^{\text{sc}}) &\leq n_b(\kappa-\log(\kappa)-1)\max_{i} \mu_i + (n-n_b)\bigg(\frac{m_b\max_{j}\nu_j}{n\kappa\min_{i,j} K_{ij}} + (m- m_b) \varepsilon^2 - \min_{i}\mu_i\\
&\qquad + \max_{i} \mu_i\log\Big(\frac{\kappa(n-n_b+ 1)\max_{i} \mu_i}{m_bK_{\min}\min_{j \in J_{\varepsilon,\kappa}}\nu_j} + \frac{n_b\kappa^2(\max_{i} \mu_i)^2}{mm_b\varepsilon^2 K_{\min}^2\min_{j \in J_{\varepsilon,\kappa}}\nu_j}\Big).
\end{align*}
Finally, by Lemma~\ref{lem:pinsker} we obtain
\begin{align*}
\norm{{\mu} -{\mu}^{\text{sc}}}^2_1 \leq & n_b(\kappa-\log(\kappa)-1)\max_{i} \mu_i + 7(n-n_b)\bigg(\frac{m_b\max_{j}\nu_j}{n\kappa\min_{i,j} K_{ij}} + (m- m_b) \varepsilon^2 - \min_{i}\mu_i\\
&+ \max_{i} \mu_i\log\Big(\frac{\kappa(n-n_b+ 1)\max_{i} \mu_i}{m_bK_{\min}\min_{j \in J_{\varepsilon,\kappa}}\nu_j} + \frac{n_b\kappa^2(\max_{i} \mu_i)^2}{mm_b\varepsilon^2K_{\min}^2\min_{j \in J_{\varepsilon,\kappa}}\nu_j}\Big).%\bigg\}^{1/2}
\end{align*}
Proof of the upper bound for $\norm{\nu - {\nu}^{\text{sc}}}^2_1$ follows the same lines as above.



\subsection{Additional experimental results}
\small
\bibliography{biblio}
\bibliographystyle{plain}

\end{document}
\section{Introduction} % (fold)
\label{sec:introduction}

Computing OT distances between pairs of probability measures or histograms, such as the earth mover's distance~\cite{werman1985,Rubner2000} and Monge-Kantorovich or Wasserstein distance~\citep{villani09optimal}, are currently generating an increasing attraction in different machine learning tasks~\cite{pmlr-v32-solomon14,kusnerb2015,pmlr-v70-arjovsky17a,ho2017}, statistics~\cite{frogner2015nips,panaretos2016,ebert2017ConstructionON,bigot2017,flamary2018WDA}, and computer vision~\cite{bonnel2011,Rubner2000,solomon2015}, among other applications~\cite{klouri17,peyre2019COTnowpublisher}.
In many of these problems, OT exploits the geometric features of the objects at hand in the underlying spaces to be leveraged in comparing probability measures.
This effectively leads to improve performance of methods that are oblivious to the geometry, for example the chi-squared distances or the Kullback-Leibler divergence.
Unfortunately, this advantage comes at the price of an enormous computational cost of solving the OT problem, that can be prohibitive in large scale applications.
For instance, the OT between two histograms with supports of equal size $n$ can be formulated as a linear programming problem that requires generally super $\bigO(n^3)$~\cite{pele2009} arithmetic operations, which is problematic when $n$ is larger than $10^3.$

A remedy to the heavy computation burden of OT lies in a prevalent approach referred as regularized OT~\cite{cuturinips13} and operates by adding an entropic regularization penalty to the original problem.  
Such a regularization guarantees a unique solution, since the objective function is strongly convex, and a greater computational stability.
Furthermore,~\cite{cuturinips13} proposed the so-called {dual of Sinkhorn divergence} as the dual of the entropic problem and noticed that finding the dual solution was equivalent to finding two diagonal matrices that made a full matrix bistochastic.
Therefore, the OT can be solved efficiently with celebrated matrix scaling algorithms, such as Sinkhorn's fixed point iteration method~\cite{sinkhorn1967,knight2008,kalantari2008}. 

% \paragraph{Related work.} 

Sinkhorn scaling for computing OT distances is a well studied problem in many recent works. 
The main idea is to improve the matrix-vector operations that are the true computational bottleneck of Sinkhorn’s algorithm. 
\cite{altschulernips17} proposed the Greenkhorn algorithm, a greedy version of Sinkhorn algorithm that selects columns and rows to be updated that most violate the constraints.
\cite{altschuler2018Nystrom} provided the Nys-Sink algorithm which is based on low-rank approximation of the cost matrix using Nystrom method.
Other classical optimization algorithms have been considered to approximate the OT, for instance accelerated gradient descent~\cite{dvurechensky18aICML,lin2019}, quasi Newton methods~\cite{blondel2018ICML,cuturi2016SIAM} and stochastic gradient descent~\cite{genevay2016stochOT,khalilabid2018}. 

% \paragraph{Our contribution.}

We give a new algorithm to approximate the regularized OT distance between discrete measures. 
Our algorithmic analysis is based on an approximate of the dual of Sinkhorn divergence by adding new constraints feasibility. 
These constraints are defined through a convex set which depends on two parameters, acting like threshold and scaling factor.
We prove that dual solution of this approximation guarantees the existence of two active indices sets for the potential variables.
These active sets are both directly linked to a priori fixed number budget of points from the supports of the given discrete measures.
We then restrict the constraints feasibility with respect to the active sets to get a ``screened'' version of the dual of Sinkhorn divergence. 
The Screenkhorn algorithm developed in this paper relies on two steps; the first one consists of an initialization step devoted to determine the active sets while the second is a constrained L-BFGS solver~\cite{zhu1997-LBFGS,cuturi2016SIAM,blondel2018ICML}. 

% \paragraph{Outline of paper.} 

The remainder of the paper is organized as follow. In Section~\ref{sec:regularized_discrete_ot} we briefly review the basic setup of regularized discrete OT. 
Section~\ref{sec:screened_dual_of_sinkhorn_divergence} contains our main contribution, that is, the Screenkhorn algorithm. 
Section~\ref{sec:analysis_of_marginal_violations} devotes to theoretical guarantees for marginal violations of Screenkhorn. 
In Section~\ref{sec:numerical_experiments} we present numerical results for the proposed algorithm, compared with the state-of-art Sinkhorn algorithm as implemented in~\cite{flamary2017pot}. 
The proofs of theoretical results are postponed to the supplementary material.

\paragraph{Notation.}

For any positive matrix $T \in \R^{n\times m}$, we define its negative entropy as $H(T) = -\sum_{i,j} T_{ij} \log(T_{ij}).$
Let $r(T) = T\mathbf 1_m \in \R^n$ and $c(T) = T^\top\mathbf 1_n \in \R^m$ denote the rows and columns sums of $T$ respectively. The coordinates $r_i(T)$ and $c_j(T)$ denote the $i$-th row sum and the $j$-th column sum of $T$, respectively.
The scalar product between two matrices denotes the usual inner product, that is $\inr{T, W} = \text{tr}(T^\top W) = \sum_{i,j}T_{ij}W_{ij},$ where $T^\top$ is the transpose of $T$. 
We write $\mathbf{1}$ (resp. $\mathbf{0}$) the vector having all coordinates equal to one (resp. zero).
$\Delta(w)$ denotes the diag operator, such that if $w \in \R^n$, then $\Delta(w) = \text{diag}(w_1, \ldots, w_n)\in \R^{n\times n}$.
For a set of indices $L=\{i_1, \ldots, i_k\} \subseteq \{1, \ldots, n\}$ satisfying $i_1 < \cdots <i_k,$ we denote the complementary set of $L$ by $L^\complement = \{1, \ldots, n\} \backslash L$. We also denote $|L|$ the cardinality of $L$.
Given a vector $w \in \R^n$, we denote $w_L= (w_{i_1}, \ldots, w_{i_k})^\top \in \R^k$ and its complementary $w_{L^\complement} \in \R^{n- k}$.  The notation is similar for matrices; given another subset of indices $S = \{j_1, \ldots, j_l\} \subseteq \{1, \ldots, m\}$ with $j_1 < \cdots <j_l,$ and a matrix $T\in \R^{n\times m}$, we use $T_{(L,S)}$, to denote the submatrix of $T$, namely the rows and columns of $T_{(L,S)}$ are indexed by $L$ and $S$ respectively.
When applied to matrices and vectors,  $\odot$ and $\oslash$ (Hadamard product and division) and exponential notations refer to elementwise operators.
Given two real numbers $a$ and $b$, we write $a\vee b = \max(a,b)$ and $a\wedge b = \min(a,b).$

% section introduction (end)